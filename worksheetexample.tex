
\documentclass[asy,breakable]{worksheet}

\begin{document}
\headerwithname{{\large\textbf{\textsc{Brown University}}} \\
  Problem Set 10 \\
  Instructor: Samuel S.\ Watson \\
  Due: 1 December 2017}{3cm}

\begin{center}
  \begin{minipage}{0.8\textwidth} 
    \textit{Print out these pages, including the additional space at the end, and complete the problems by hand. Then use Gradescope to scan and upload the entire packet by 18:00 on the due date.}
  \end{minipage}
\end{center}

\begin{problem} Sketch the vector fields
$\displaystyle{\mathbf{F}_1(x,y) = \frac{y\, \mathbf{i} - x\,
  \mathbf{j}}{\sqrt{x^2+y^2}}}$ and $\displaystyle{\mathbf{F}_2(x,y) = \frac{y\, \mathbf{i} - x\,
  \mathbf{j}}{x^2+y^2}}$.
\end{problem}

\sol[height = 8cm] 

\begin{solution}[height = 8cm] 
  \begin{center} 
  \begin{asy} 
    size(6cm);
    import vectorfield; 
pair a = (-2,-2);
pair b = (2,2);
real f(real x, real y) {return x^2+y^2  == 0 ? 10 : min(10,1/sqrt(x^2+y^2));}
path vector(pair z) {real x = z.x; real y = z.y; return
  (0,0)--(y*f(x,y), -x*f(x,y));}
draw((a.x,0)--(b.x,0),Arrow(4));
draw((0,a.y)--(0,b.y),Arrow(4));
add(vectorfieldmid(vector,a,b,0.5*blue));
\end{asy}
\quad
\begin{asy} 
  size(6cm);
  import vectorfield; 
pair a = (-2,-2);
pair b = (2,2);
real f(real x, real y) {return x^2+y^2  == 0 ? 10 : min(10,1/(x^2+y^2));}
path vector(pair z) {real x = z.x; real y = z.y; return
  (0,0)--(y*f(x,y), -x*f(x,y));}
draw((a.x,0)--(b.x,0),Arrow(4));
draw((0,a.y)--(0,b.y),Arrow(4));
add(vectorfieldmid(vector,a,b,0.5*blue));
\end{asy}
\end{center}
\end{solution}

\begin{problem} Find $\int_C (x+2y) \,dx + x^2 \, dy$ where $C$ is the
  concatenation of the line segment from $(0,0)$ to $(2,1)$ and the
  line segment from $(2,1)$ to $(3,0)$. (Note: the notation
  $(x+2y) \,dx + x^2 \, dy$ is another way of writing
  $\mathbf{F}\cdot d\mathbf{r}$, where
  $\mathbf{F} = \langle x+2y, x^2 \rangle$.)
\end{problem}

\solfinalanswer

\begin{solution}[height fill] 
  The first segment can be parametrized as $\mathbf{r}(t) = \langle
  2t,t \rangle$ as
  $t$ ranges from 0 to 1, so the integral along that segment is
  \[
    \int_0^1 \langle 2t + 2t, 4t^2 \rangle \cdot \langle 2, 1 \rangle
    \, dt  = \int_0^1 8t + 4t^2 \, dt = 4 + \frac{4}{3} =
    \frac{16}{3}. 
  \]
  The second segment can be parametrized as $\langle 2 + t, 1- t
  \rangle$ as $t$ ranges from 0 to 1, so the integral along that
  segment is
    \[
    \int_0^1 \langle 2+t + 2(1-t), (2+t)^2\rangle \cdot \langle 1, -1 \rangle
    \, dt  = \int_0^1 4-t - (2+t)^2 dt = -\frac{17}{6}. 
  \]
  So the integral along the concatenation of the two segments is
  $\displaystyle{\frac{16}{3} - \frac{17}{6} }= \frac{5}{2}$. 

  \finalanswer[\Large $\displaystyle{\frac{5}{2}}$]
\end{solution}

\end{document}
