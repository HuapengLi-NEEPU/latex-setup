
\documentclass[julia,solutions]{worksheet}

\usepackage{lipsum} 

\begin{document}

\headerwithname{
    The \texttt{worksheet} document class \\ 
    Samuel S.\ Watson \\ \today
  }{
    3 cm
  }

The document class \texttt{worksheet} is designed for formatting problems and solutions using the \texttt{tcolorbox} package. It supports formatting a problems-only and solutions-included document from a single source file. The \texttt{solutionbox} environment is for blank solution space, while the \texttt{solution} environment is for the actual solutions. 

\section{Basic commands}

The main environments are \texttt{problem} and \texttt{solution}. Standard \texttt{tcolorbox} options, like \texttt{height = 5cm} or \texttt{height fill} can be passed to the tcolorbox.

Typically one would have a \texttt{problem} environment followed by a \texttt{solutionbox} environment followed by a \texttt{solution} environment. Only one of the latter two will be displayed, according to whether the \texttt{solutions} document option is invoked (that is \texttt{\textbackslash documentclass[solutions]\{worksheet\}} will show solutions, while \texttt{\textbackslash documentclass[]\{worksheet\}} will show empty boxes). 

\begin{verbatim}
\begin{problem}
  $\vdots$
\end{problem}
\end{verbatim}

\begin{problem}
  $\vdots$
\end{problem}

\begin{verbatim} 
\begin{solution}[height = 5cm] 
  $\vdots$
\end{solution}
\end{verbatim}

\begin{solution}[height = 5cm] 
  $\vdots$
  
  Equations, like \eqref{eq:1} below, are numbered on a per-problem basis. 
  \begin{equation} \label{eq:1}
    \int \frac{1}{1+x^2} \, \d x = \arctan x. 
  \end{equation}
\end{solution}

You can put a final answer box in the lower right-hand corner with

\begin{verbatim}
\finalanswer[$\frac{5}{2}$]
% or 
\finalanswer[$\frac{5}{2}$][3cm][4cm]
\end{verbatim} 

or, for an empty final answer box, just \texttt{\textbackslash finalanswer}. The macro \texttt{\textbackslash solfinalanswer} gives a height-filling empty solution box with an empty final answer box. 

\section{Header commands}

The header is produced with the \texttt{\textbackslash header} command. A name can be included via \texttt{\textbackslash headerwithname}, which takes two arguments: the header content and the width of the name box. So the header above was produced with
\begin{verbatim} 
\headerwithname{
    The \texttt{worksheet} document class \\ 
    Samuel S.\ Watson \\ \today
  }{
    3 cm 
  }
\end{verbatim}

\section{Other commands} 

We can also make a problem using the \texttt{\textbackslash pb} shortcut

\begin{verbatim} 
\pb{Another example problem}
\end{verbatim}

\pb{
  Another example problem
}

Likewise, we can produce a solutionbox using \texttt{\textbackslash sol}, which takes an optional argument of tcolorbox options, like \texttt{height = 5cm}. 

We can put two problems side by side using the \texttt{problempair} environment (in which one must place two \texttt{problem} environments) or using \texttt{twoproblems}. 

\begin{verbatim}
\twoproblems{first problem contents}{second problem contents}
\end{verbatim}
\twoproblems{first problem contents}{second problem contents}

We can also put two empty solution boxes side by side. The argument is any list of \texttt{tcolorbox} options. 
\begin{verbatim}
\twosolutions{height = 5cm}
\end{verbatim}
  \begin{tcbraster}[
      raster columns=2,
      height = 5cm 
    ]
    \begin{solution}
    \end{solution}
    \begin{solution}
    \end{solution}
  \end{tcbraster}

  \vspace{12pt}

\section{Inset figures} 

The \texttt{wrapfigure} package doesn't play so nicely with tcolorboxes. But \texttt{worksheet.cls} has its own \texttt{insetfigure} command which juxtaposes boxes containing its first and second arguments, with alignment specified by the optional third argument (\texttt{c} for center, \texttt{t} for top, and \texttt{b} for bottom, \texttt{t} by default) and spacing specified by the optional fourth argument (\texttt{16pt} by default). Here's an example:

\begin{verbatim}
\insetfigure{
  \lipsum[1]
}{
  \includegraphics[width=3cm]{example-grid-100x100pt}
}
\lipsum[2-3]
\end{verbatim}
\insetfigure{
  \lipsum[1]
}{
  \includegraphics[width=3cm]{example-grid-100x100pt}
}
\lipsum[2-3]

\section{Solutions}

The accompanying Python script \texttt{processsolutions.py} can be used to simultaneously produce a pdf with solutions and one without. The only caveat here is that the square brackets in the \texttt{documentclass} command must be present, even if they're empty. 

\section{Julia integration}

The \texttt{pythontex} package can be used to weave Julia code into the worksheets. Combined with SymPy, which can be accessed from Julia, this can be pretty powerful. For example: 

%MATRIX EXAMPLE------------
\begin{verbatim} 
\begin{jlcode}
  A = rand(-5:5,5,10)
  x = rand(-5:5,10)
\end{jlcode}

\[
  \jltex{A} \jltex{x} = \jltex{A*x}. 
\]
\end{verbatim}

\begin{jlcode}
  A = rand(-5:5,5,10)
  v = rand(-5:5,10)
\end{jlcode}

\[
  \jltex{A} \jltex{v} = \jltex{A*v}. 
\]
% --------------------------

% Symbolic example ---------
\begin{verbatim}
\begin{jlcode}
  I = Integral(exp(x^2),x)
\end{jlcode}

\[
  \jltex{I} = \jltex{doit(I)}. 
\]
\end{verbatim}

\begin{jlcode}
  I = Integral(exp(x^2),x)
\end{jlcode}

\[
  \jltex{I} = \jltex{doit(I)}. 
\]
%--------------------------

The Julia function \texttt{p}, which simply returns its argument in string form, also increments a global Julia variable \texttt{POINTS}, which can be used to keep a running total of the points assigned so far.

\begin{problem}[title = Problem \itm (\jltex{p(5)} points)]
  This is the first problem. 
\end{problem}

\begin{problem}[title = Problem \itm (\jltex{p(8)} points)]
  This is the second problem. 
\end{problem}

Total points so far: \jltex{POINTS}

You can use \texttt{Plots.jl} with its \texttt{pgfplots} backend to
generate figures which can then be included. 

\begin{jlcode}
  using Plots
  pgfplots()
  plot(x^2,title="The squaring function",legend=:none)
  savefig("myplot.pdf")
\end{jlcode}

\begin{center}
  \jl{"\\includegraphics{myplot}"}
\end{center}

\end{document}
